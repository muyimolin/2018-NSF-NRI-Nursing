\section*{Human Subjects Protection}

The University of Illinois at Chicago has an Institutional Review Board (IRB) that reviews all
research involving the use of human subjects
%, before the research can be initiated 
(UIC Human Subjects Assurance No.: FWA00000083, effective
through 03/05/17). 
All PIs are familiar with IRB policies, including
fulfilling the required continuing  training  in human subjects protection. 


%An existing UIC IRB protocol, \#~2012-0823, grants this project  exemption according to category~4 as designated in the Common Rule (Federal Policy for the Protection of Human Subjects,  45 CFR 690). Exemption has been granted for access to  the de-identified discharge notes and hospital nursing documentation, since this data  does not include  any of the 18 HIPAA elements. Hence,  de-identification enables collaboration without risking the loss of identifiable Personal Health Information (PHI) data.  


We will submit an IRB protocol to cover the collection of Datasets~1 through 3, and final evaluation.  We foresee the IRB will review the  protocol
under expedited procedure, since it is minimal risk. The protocol employs text messaging between a patient, and either a human health coach, or  the virtual health coach (under human health coach supervision). The focus of the messages is healthy eating and physical activities behavior.  Survey methods will be employed.

Patients will be reminded that text messages are sent in an automated fashion by computer. Thus, any replies may not be read in a timely fashion. Text messaging by patients should not be considered an Emergency Response System. Furthermore, they will be told and reminded that urgent health questions should be directed to their physician and not sent in a text message.

mytapp includes multiple security measures. Limited subject information will be available on mytapp, such as name and study ID number, and telephone number stored in an encrypted PostgreSQL database. We utilize TLS (Transport Layer Security) with encryption for all mytapp interactions and interactions with Twilio. All information from Twilio to mytapp is validated against embedded digital signatures to ensure authenticity and prevent “spoofing.” A “PaperTrail” application provides a continuous audit trail to monitor mytapp use and all message activity. Message data downloaded from mytapp will be stripped of identifiers (phone number, name, etc.) but include subject ID for analyses.

 When reporting on the structure of the SMS interactions, or on the results from the  evaluation,  all names and other potentially identifying features are changed to protect the confidentiality of the patients, their families and friends, their care-providers.\OUT{, and any other participants} In short, no individual information will ever be revealed. In any \OUT{subsequent} presentations or publications, data will be presented either in aggregate form or in short excerpts which serve to illustrate  typical  interactions. All identifiable information will be removed from those  excerpts. At the end of the project we will  release the de-identified SMS from subjects who have provided additional  consent to  public release of their data. 

%\subsection*{Potential Risks}

 \paragraph{Potential Risks.} As concerns privacy and confidentiality, 
 the de-identified SMS do not contain any of the 18 HIPAA elements. Small risks include: (1) loss of confidentiality; (2) inaccurate or misinterpretation of accurate information; (3) annoyance of SMS contact. It is unlikely that these risks would result in serious consequences. Overall benefits of attained improved health behaviors and knowledge gained by study outweigh potential risks involved.  
 
 The de-identified data will be kept on password-locked computers, including one of the servers  in PI Ziebart's   laboratory,  and one of the servers in PI Di~Eugenio's  laboratory. Both laboratories are in the Science and
Engineering Laboratories building. Each laboratory   is locked and only the PIs and their
graduate assistants have keys. The data will be
 accessible only to the PIs and their graduate assistants. 
The data will be stored on different 
servers  because software tools that are needed to process the data are not available on every server, e.g. because of licensing issues.


  
%\subsection*{Recruitment and informed consent}

\paragraph{Recruitment and informed consent.} A total of 200 overweight or obese  patients, 50 per year, will be recruited  from University of Illinois Health. 
%(UI Health). 
%and focus on healthy eating and physical activity behaviors. 
%UI Health includes both inpatient and outpatient facilities
%serving a diverse population in Chicago (approximately 50\% African-American, 25\% Latino, and the rest other racial/ethnic groups). 
UI Health has 19 ambulatory centers staffed by Family Medicine and Internal Medicine providers and includes Federally Qualified Health Centers (FQHCs). An on-site research assistant at one or two locations will identify patients potentially eligible, evaluating the following criteria: (1) written fluency in English, (2) mobile phone with text messaging plan, (3) age 21-65, and (4) overweight or obese (body mass index greater than 25 kg/m\textsuperscript{2}). Exclusion criteria include: (1) unable to verbalize comprehension of study protocol or impaired decision making (e.g., dementia), (2) household member already participating in study, (3) pregnancy or trying to get pregnant, and (4) disability or other multimorbidity that prevents physical activity. \OUT{We will recruit 50 participants over 6 months;} Each participant will be provided with a pedometer and/or scale.

\OUT{the participating clinical sites, a trained} The on-site research assistant will \OUT{be available to} discuss the risks and benefits of the study and answer any questions. The following will be explained: (1)  purpose; (2) financial compensation for completing the study; (3) types of data collected; (4) virtual and human coach interventions; (5) measures to ensure confidentiality of data collected; and  (6) timeline. Regarding data collection, subjects will be made aware of the methodology, including surveys and text messaging. They will also be informed that, if they provide separate  consent, their de-identified data will be released to other research groups at the end of the project. Participants are informed that they would be responsible for text messaging costs if they do not maintain a text messaging plan. The consent form  will be read and explained in detail and include the described information. Written informed consent will then be obtained from individuals who wish to participate. 

	
 
%\subsection*{Inclusion of women, minorities and children}
\paragraph{Inclusion of women, minorities and children.}
We plan to recruit only adults over age 21 who are capable of providing consent. The health coach communications will focus on approaches to adult health behaviors, and not necessarily applicable for children, though this may be pursued in future study. \OUT{For the present study, neither minors nor persons with cognitive impairments will be recruited.}  We will recruit women and minorities to ensure that our human subjects are representative of the local general population of the city in which our institution is located.  The UI hospital patient population is currently approximately 50\% African-American, 25\% Latino, and the rest other racial/ethnic groups. Based on previous studies, we anticipate twice as many women participants as men.

	
  
%\subsection*{Planned procedures to protect against or minimize potential risks}

\paragraph{Planned procedures to protect against or minimize potential risks.} The SMS will be de-identified, and the  de-identified SMS  will be kept in  dedicated, protected directories on secure servers.   Each office/ laboratory is locked and only the PIs and their graduate assistants have keys. 
When the results of the research are published or discussed in
conferences, only aggregated data, or short excerpts, will be disclosed.  %The de-identified SMS  will not be published. 


It is possible that subjects may receive and act on inaccurate information or misinterpret accurate information given the novel communication methods utilized. \OUT{To address this, coaches receive training and ongoing evaluation to reduce the risk of providing misinformation.} Dr. Gerber will routinely monitor 10\% of communications for quality and fidelity. To address any possibility for subject concern regarding information received, we will encourage them to seek help from their primary care physician if they are uncomfortable with any of the information provided by health coaches.

There is a small risk of breach of confidentiality. All research staff, including health coaches, will have formal IRB human subject protection and HIPAA training. Project staff who have access to the data will not have access to subject names. No clinical staff who have direct contact with the subjects will have access to research data. No personal information will be stored on the servers. 

Subjects will receive \OUT{regular scheduled} text messages during the study. Participants who no longer wish to receive \OUT{reminders or text messages} them may reply with “STOP.” Twilio automatically will block mytapp from sending messages if this is received; thus, messages will be immediately discontinued without requiring contact with research team. However, patients have the option to re-subscribe with “START” again and resume messages. mytapp also color-codes blocked messages as “red” which will be easily detected by health coaches. This allows the health coaches to identify problems including loss of mobile service.

%The PIs  will meet bi-weekly. During these meetings, they will also discuss the data collected, review the progress of various studies, and ensure that all potential risks to human subjects are minimized.
