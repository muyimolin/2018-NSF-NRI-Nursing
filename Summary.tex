\pagebreak

\begin{center}
	{\Large \bf Intelligent Mediating Robot---Lifelong Learning and Fast Intent Forecasting in Simultaneous Multi-lateral Physical Human-robot Interaction}\\
    \vspace{4pt}
% 	\renewcommand{\baselinestretch}{1}
   	{\large PI: Zhi Li (Worcester Polytechnic Institute)\\
    Co-PI: Brian Ziebart (University of Illinois at Chicago), Jie Fu (Worcester Polytechnic Institute)}
   	% \vspace{4pt}
    % {\large Worcester Polytechnic Institute}
\end{center}

\vspace{1 em}

\paragraph*{\Large Project Summary} 
This project aims to endow tele-robot systems with the capabilities of lifelong learning and fast intent forecasting.
Through robot-mediated interactions between expert teleoperators and remote human nurses/patients, the proposed system will develop its motion intelligence to assist novice teleoperators in performing patient-caring tasks that require fine motor skills and intimate human-robot interactions.
Our proposed \textbf{lifelong robot learning approach} enables robots to learn high-level task structures and low-level motion skills, and adapt these acquired capabilities to new task contexts according to user preferences.
We also propose a \textbf{systematic approach for fast intent forecasting} to infer the joint rewards and constraints underlying collaborative task decomposition and motion coordination, and to develop shared-autonomous control strategies to reduce the operation and learning effort of novice teleoperators.
Our proposed approaches will be implemented on a mobile humanoid nursing robot platform and evaluated through user studies.
The framework of multi-lateral physical interaction simultaneously engages a robot learner with human teachers in various roles, so that the robot can acquire contextual knowledge and skills, explore how to best synergize its physical capabilities with its various level of motion intelligence, and continuously evolve its motion intelligence through long-term learning.

% Through robot-mediated interactions, develops robot motion intelligence in a multi-agent, highly-interactive context. 



% to navigate in cluttered human environments and perform a wide variety of dexterous manipulation tasks with minimal human control. Our key idea is to develop a unified framework for lifelong learning and fast, context-based intent and in simultaneous multi-lateral physical human-robot interactions. In such scenarios, the nursing robot participates in a patient-caring task, while learning when and how to intervene in the robot-mediated collaboration between its remote teleoperator and on-site human partners. By observing human experts, the nursing robot will also establish hierarchical knowledge of natural coordinated human motions and human-human interactions, and metrics for evaluating task performance and motion capabilities. Such motion knowledge and metrics will be used to evaluate the level of skills of novice teleoperators, patients and partner nurses, and adjust the level of assistance provided to maintain nursing task performance and fluency of human-robot collaboration. 

  
\vspace{0.5 em}

% \paragraph*{\Large Intellectual Merit}
% Our proposal aims to address the need for \textit{\textbf{customizable robot motion intelligence}}, and to \textit{\textbf{lower the barriers}} for medical personnel to synergize with tele-robotic technologies. Our proposed framework enables a tele-robotic system to develop and evolve its contextual motion intelligence through lifelong interactions with human experts, and apply its motion intelligence to provide user-adaptive assistance to reduce learning and operation effort for novice teleoperators.   

\paragraph*{\Large Intellectual Merit}
The use of teleoperated shared-autonomous robotic systems is a promising avenue for addressing several challenges in society including aging and highly infectious diseases. However, the research community lacks the fundamental knowledge to build \textit{\textbf{customizable robot motion intelligence}}, which is the key to \textit{\textbf{lowering the barriers}} for medical personnel to synergize with telerobotic technologies, and these robots more ubiquitous in this setting. To this end, the proposed work will develop theories and general methods that enable an autonomous system to gain and evolve its contextual motion intelligence at both high-level task planning and low-level motion planning through lifelong collaboration and interactions with human experts. Through learning from robot-mediated interactions, we provide a paradigm that can efficiently transfer task knowledge and motion skills between humans and robots, and unite human capabilities of decision making, task performance evaluation and correction, with the robot's accurate motion control and perception.
Our work will not only unite various learning paradigms to build coherent motion intelligence for robotic systems, but also leverage the learned motion intelligence to develop user-adaptive assistance and reduce training and operation efforts for novice teleoperators. 

\vspace{0.5 em}

\paragraph*{\Large Broader Impacts}
This project envisions broader impacts on a wide range of mobile humanoid robots for medical, industrial, and social service tasks. Our research efforts will enable these robots to interact simultaneously and physically with both the teleoperator and end users, and reconcile their intents to achieve natural, fluent, and intimate collaboration.
% PM: I changed this to singular "barrier" to make the claim a bit less grand? Feel free to change back if desired.
We aim to remove a major barrier that prevents robots from integrating into human society as capable and socially acceptable peers. By improving the usability of dexterous robotic manipulators under direct teleoperation and shared-autonomous control, this project may also lead to improved availability of healthcare, industrial, and social service labor, and provide surrogates for military and medical personnel for tedious, repetitive, and dangerous tasks. It will lead towards affordable robotic solutions for hospital and home care that can provide long-term assistance to aging and disabled populations. Our research will also synergize with graduate and undergraduate education for students from engineering, medical, and nursing schools, and will actively engage K-12 students and the general public. 

\vspace{2 em}
\noindent
\textbf{Keywords} --- Customizability, Lowering Barriers, Learning, Human-Robot Interaction, Medical